% Options for packages loaded elsewhere
\PassOptionsToPackage{unicode}{hyperref}
\PassOptionsToPackage{hyphens}{url}
%
\documentclass[
]{article}
\usepackage{amsmath,amssymb}
\usepackage{lmodern}
\usepackage{ifxetex,ifluatex}
\ifnum 0\ifxetex 1\fi\ifluatex 1\fi=0 % if pdftex
  \usepackage[T1]{fontenc}
  \usepackage[utf8]{inputenc}
  \usepackage{textcomp} % provide euro and other symbols
\else % if luatex or xetex
  \usepackage{unicode-math}
  \defaultfontfeatures{Scale=MatchLowercase}
  \defaultfontfeatures[\rmfamily]{Ligatures=TeX,Scale=1}
\fi
% Use upquote if available, for straight quotes in verbatim environments
\IfFileExists{upquote.sty}{\usepackage{upquote}}{}
\IfFileExists{microtype.sty}{% use microtype if available
  \usepackage[]{microtype}
  \UseMicrotypeSet[protrusion]{basicmath} % disable protrusion for tt fonts
}{}
\makeatletter
\@ifundefined{KOMAClassName}{% if non-KOMA class
  \IfFileExists{parskip.sty}{%
    \usepackage{parskip}
  }{% else
    \setlength{\parindent}{0pt}
    \setlength{\parskip}{6pt plus 2pt minus 1pt}}
}{% if KOMA class
  \KOMAoptions{parskip=half}}
\makeatother
\usepackage{xcolor}
\IfFileExists{xurl.sty}{\usepackage{xurl}}{} % add URL line breaks if available
\IfFileExists{bookmark.sty}{\usepackage{bookmark}}{\usepackage{hyperref}}
\hypersetup{
  pdftitle={Violence against women (global)},
  pdfauthor={Gemittarius},
  hidelinks,
  pdfcreator={LaTeX via pandoc}}
\urlstyle{same} % disable monospaced font for URLs
\usepackage[margin=1in]{geometry}
\usepackage{color}
\usepackage{fancyvrb}
\newcommand{\VerbBar}{|}
\newcommand{\VERB}{\Verb[commandchars=\\\{\}]}
\DefineVerbatimEnvironment{Highlighting}{Verbatim}{commandchars=\\\{\}}
% Add ',fontsize=\small' for more characters per line
\usepackage{framed}
\definecolor{shadecolor}{RGB}{248,248,248}
\newenvironment{Shaded}{\begin{snugshade}}{\end{snugshade}}
\newcommand{\AlertTok}[1]{\textcolor[rgb]{0.94,0.16,0.16}{#1}}
\newcommand{\AnnotationTok}[1]{\textcolor[rgb]{0.56,0.35,0.01}{\textbf{\textit{#1}}}}
\newcommand{\AttributeTok}[1]{\textcolor[rgb]{0.77,0.63,0.00}{#1}}
\newcommand{\BaseNTok}[1]{\textcolor[rgb]{0.00,0.00,0.81}{#1}}
\newcommand{\BuiltInTok}[1]{#1}
\newcommand{\CharTok}[1]{\textcolor[rgb]{0.31,0.60,0.02}{#1}}
\newcommand{\CommentTok}[1]{\textcolor[rgb]{0.56,0.35,0.01}{\textit{#1}}}
\newcommand{\CommentVarTok}[1]{\textcolor[rgb]{0.56,0.35,0.01}{\textbf{\textit{#1}}}}
\newcommand{\ConstantTok}[1]{\textcolor[rgb]{0.00,0.00,0.00}{#1}}
\newcommand{\ControlFlowTok}[1]{\textcolor[rgb]{0.13,0.29,0.53}{\textbf{#1}}}
\newcommand{\DataTypeTok}[1]{\textcolor[rgb]{0.13,0.29,0.53}{#1}}
\newcommand{\DecValTok}[1]{\textcolor[rgb]{0.00,0.00,0.81}{#1}}
\newcommand{\DocumentationTok}[1]{\textcolor[rgb]{0.56,0.35,0.01}{\textbf{\textit{#1}}}}
\newcommand{\ErrorTok}[1]{\textcolor[rgb]{0.64,0.00,0.00}{\textbf{#1}}}
\newcommand{\ExtensionTok}[1]{#1}
\newcommand{\FloatTok}[1]{\textcolor[rgb]{0.00,0.00,0.81}{#1}}
\newcommand{\FunctionTok}[1]{\textcolor[rgb]{0.00,0.00,0.00}{#1}}
\newcommand{\ImportTok}[1]{#1}
\newcommand{\InformationTok}[1]{\textcolor[rgb]{0.56,0.35,0.01}{\textbf{\textit{#1}}}}
\newcommand{\KeywordTok}[1]{\textcolor[rgb]{0.13,0.29,0.53}{\textbf{#1}}}
\newcommand{\NormalTok}[1]{#1}
\newcommand{\OperatorTok}[1]{\textcolor[rgb]{0.81,0.36,0.00}{\textbf{#1}}}
\newcommand{\OtherTok}[1]{\textcolor[rgb]{0.56,0.35,0.01}{#1}}
\newcommand{\PreprocessorTok}[1]{\textcolor[rgb]{0.56,0.35,0.01}{\textit{#1}}}
\newcommand{\RegionMarkerTok}[1]{#1}
\newcommand{\SpecialCharTok}[1]{\textcolor[rgb]{0.00,0.00,0.00}{#1}}
\newcommand{\SpecialStringTok}[1]{\textcolor[rgb]{0.31,0.60,0.02}{#1}}
\newcommand{\StringTok}[1]{\textcolor[rgb]{0.31,0.60,0.02}{#1}}
\newcommand{\VariableTok}[1]{\textcolor[rgb]{0.00,0.00,0.00}{#1}}
\newcommand{\VerbatimStringTok}[1]{\textcolor[rgb]{0.31,0.60,0.02}{#1}}
\newcommand{\WarningTok}[1]{\textcolor[rgb]{0.56,0.35,0.01}{\textbf{\textit{#1}}}}
\usepackage{graphicx}
\makeatletter
\def\maxwidth{\ifdim\Gin@nat@width>\linewidth\linewidth\else\Gin@nat@width\fi}
\def\maxheight{\ifdim\Gin@nat@height>\textheight\textheight\else\Gin@nat@height\fi}
\makeatother
% Scale images if necessary, so that they will not overflow the page
% margins by default, and it is still possible to overwrite the defaults
% using explicit options in \includegraphics[width, height, ...]{}
\setkeys{Gin}{width=\maxwidth,height=\maxheight,keepaspectratio}
% Set default figure placement to htbp
\makeatletter
\def\fps@figure{htbp}
\makeatother
\setlength{\emergencystretch}{3em} % prevent overfull lines
\providecommand{\tightlist}{%
  \setlength{\itemsep}{0pt}\setlength{\parskip}{0pt}}
\setcounter{secnumdepth}{-\maxdimen} % remove section numbering
\ifluatex
  \usepackage{selnolig}  % disable illegal ligatures
\fi

\title{Violence against women (global)}
\author{Gemittarius}
\date{21/06/2021}

\begin{document}
\maketitle

Project final report should be written in \texttt{R\ Markdown} and
.\texttt{pdf} output (or \texttt{.html} output) should be at most
25-pages long. All team members should be involved in the preparation of
the project final report. Project final report is due within during
\textbf{Final exam week}. Project final reports should present
appropriate details on the following topics (please note that other than
Title, Team members, and Conclusion sections, order does not need to
necessarily be in the order seen below):

\begin{itemize}
\tightlist
\item
  Project title page (including project title and team members)
\item
  Project Description
\item
  Project goal \& social problem addressed
\item
  Project data \& access to data
\item
  Actions taken (data importing, cleaning, reshaping, exploring,
  visualizing etc) along with results
\item
  Results and Discussion
\item
  Conclusion
\item
  Please provide url address of your project deployed at \texttt{GitHub}
  pages.
\item
  References
\end{itemize}

Project final report should be prepared collaboratively.

\textbf{Cautionary notes}: 1) Project interim reports will not be
accepted after deadline. 2) You should download \texttt{R\ Project} on
\texttt{GitHub} to your local computer, do the changes as needed, delete
all the files you have not used to produce your proposal, then render
the \texttt{.Rmd} file to \texttt{.pdf} (or \texttt{.html} output) and
finally commit and push all the required files (including \texttt{.Rmd}
files) by June 21, 2021 23:59 via
\href{https://github.com/MAT381E}{GitHub Classroom of MAT381E
organization}.

\newpage

\hypertarget{section}{%
\subsection{}\label{section}}

\hypertarget{section-1}{%
\subsection{}\label{section-1}}

\begin{figure}
\centering
\includegraphics{https://pbs.twimg.com/media/EwDeeNRXIAoNHn1?format=jpg\&name=small}
\caption{``\emph{Image from WHO}''}
\end{figure}

\hypertarget{section-2}{%
\subsection{}\label{section-2}}

\hypertarget{section-3}{%
\subsection{}\label{section-3}}

\hypertarget{team-members}{%
\section{Team Members}\label{team-members}}

\hypertarget{aylin-suxfcmer-090160324}{%
\subsection{Aylin Sümer 090160324}\label{aylin-suxfcmer-090160324}}

\hypertarget{tuux11frulgazi-avat-090160344}{%
\subsection{Tuğrulgazi Avat
090160344}\label{tuux11frulgazi-avat-090160344}}

\newpage  

\hypertarget{section-4}{%
\subsection{}\label{section-4}}

Violence against women is a problem that should be perceived not only
locally but also globally. The UN describes violence against women and
girls (VAWG) as: `` one of the most widespread, persistent, and
devastating human rights violations in our world today. It remains
largely unreported due to the impunity, silence, stigma, and shame
surrounding it.''

The data was taken from a survey of men and women in African, Asian, and
South American countries, exploring the attitudes and perceived
justifications given for committing acts of violence against women.

In this research both subjects and the objects of the violence are
included. The data also explores different sociodemographic groups that
the respondents belong to, including: Education Level, Marital status,
Employment, and Age group. The data reveals insights into some of the
attitudes and assumptions that prevent progress in the global campaign
to end violence against women and girls, based on a representative
sample of each country.

\emph{\href{https://data.world/login?next=\%2Fmakeovermonday\%2F2020w10\%2Fworkspace\%2Ffile\%3Ffilename\%3D20200306\%2BData\%2BInternational\%2BWomen\%2527s\%2BDay\%2BViz5\%2BLaunch.csv}{Data}
can be found in this link (it requires singing up for free in order to
see it)}

\textbf{GOAL:}

Our main purpose is to show that violence against women and girls is
never acceptable or justifiable by demonstrating both numerical and
categorical variables. We will be analyzing the percentages of people
surveyed in the relevant group who agree with the question.

\newpage

\begin{Shaded}
\begin{Highlighting}[]
\FunctionTok{library}\NormalTok{(tidyverse)}
\FunctionTok{library}\NormalTok{(readxl)}
\FunctionTok{library}\NormalTok{(dplyr)}
\FunctionTok{library}\NormalTok{(readr)}
\end{Highlighting}
\end{Shaded}

\begin{verbatim}
##   RecordID     Country Gender Demographics.Question
## 1        1 Afghanistan      F        Marital status
## 2        1 Afghanistan      F             Education
## 3        1 Afghanistan      F             Education
## 4        1 Afghanistan      F             Education
## 5        1 Afghanistan      F        Marital status
## 6        1 Afghanistan      F            Employment
##          Demographics.Response                  Question Survey.Year Value
## 1                Never married ... if she burns the food  01/01/2015    NA
## 2                       Higher ... if she burns the food  01/01/2015  10.1
## 3                    Secondary ... if she burns the food  01/01/2015  13.7
## 4                      Primary ... if she burns the food  01/01/2015  13.8
## 5 Widowed, divorced, separated ... if she burns the food  01/01/2015  13.8
## 6            Employed for kind ... if she burns the food  01/01/2015  17.0
\end{verbatim}

\begin{verbatim}
##   RecordID     Country Gender Demographics.Question Demographics.Response
## 1        1 Afghanistan      F                   Age                 15-24
## 2        1 Afghanistan      F                   Age                 25-34
## 3        1 Afghanistan      F                   Age                 35-49
## 4        1 Afghanistan      M                   Age                 25-34
## 5        1 Afghanistan      M                   Age                 35-49
## 6        1 Afghanistan      M                   Age                 15-24
##                    Question Survey.Year Value
## 1 ... if she burns the food  01/01/2015  17.3
## 2 ... if she burns the food  01/01/2015  18.2
## 3 ... if she burns the food  01/01/2015  18.8
## 4 ... if she burns the food  01/01/2015   8.2
## 5 ... if she burns the food  01/01/2015   8.6
## 6 ... if she burns the food  01/01/2015   9.4
\end{verbatim}

We can see that age is repeating itself by the change in the response of
our basic question.

\begin{verbatim}
##   Demographics.Response   n
## 1                 15-24 840
## 2                 25-34 840
## 3                 35-49 840
\end{verbatim}

\begin{verbatim}
##                                  Question    n
## 1    ... for at least one specific reason 2100
## 2              ... if she argues with him 2100
## 3               ... if she burns the food 2100
## 4 ... if she goes out without telling him 2100
## 5        ... if she neglects the children 2100
## 6 ... if she refuses to have sex with him 2100
\end{verbatim}

So, we see that we have exactly 6 different answer, and 3 different age
group. Let us see how many different demographics question we have.

\begin{verbatim}
##   Demographics.Question    n
## 1                   Age 2520
## 2             Education 3360
## 3            Employment 2520
## 4        Marital status 2520
## 5             Residence 1680
\end{verbatim}

First and most important thing is that we need to make a gender study
upon this data. So, We need to split our data into 2, such as the person
who answered is female nor male.

\begin{verbatim}
##   RecordID     Country Gender Demographics.Question
## 1        1 Afghanistan      F        Marital status
## 2        1 Afghanistan      F             Education
## 3        1 Afghanistan      F             Education
## 4        1 Afghanistan      F             Education
## 5        1 Afghanistan      F        Marital status
## 6        1 Afghanistan      F            Employment
##          Demographics.Response                  Question Survey.Year Value
## 1                Never married ... if she burns the food  01/01/2015    NA
## 2                       Higher ... if she burns the food  01/01/2015  10.1
## 3                    Secondary ... if she burns the food  01/01/2015  13.7
## 4                      Primary ... if she burns the food  01/01/2015  13.8
## 5 Widowed, divorced, separated ... if she burns the food  01/01/2015  13.8
## 6            Employed for kind ... if she burns the food  01/01/2015  17.0
\end{verbatim}

\begin{verbatim}
##   RecordID     Country Gender Demographics.Question
## 1        1 Afghanistan      M        Marital status
## 2        1 Afghanistan      M             Education
## 3        1 Afghanistan      M             Residence
## 4        1 Afghanistan      M            Employment
## 5        1 Afghanistan      M             Education
## 6        1 Afghanistan      M        Marital status
##          Demographics.Response                  Question Survey.Year Value
## 1                Never married ... if she burns the food  01/01/2015    NA
## 2                       Higher ... if she burns the food  01/01/2015   4.5
## 3                        Urban ... if she burns the food  01/01/2015   4.6
## 4                   Unemployed ... if she burns the food  01/01/2015   5.2
## 5                      Primary ... if she burns the food  01/01/2015   6.3
## 6 Widowed, divorced, separated ... if she burns the food  01/01/2015   6.3
\end{verbatim}

We have 5 different case, so we may actually divide our data through
demographics questions.

\textbf{For Female;}

\begin{Shaded}
\begin{Highlighting}[]
\NormalTok{  Female\_Residence }\OtherTok{\textless{}{-}} \FunctionTok{filter}\NormalTok{(data\_Female, Demographics.Question }\SpecialCharTok{==} \StringTok{"Residence"}\NormalTok{)}
  \FunctionTok{view}\NormalTok{(Female\_Residence)}
\end{Highlighting}
\end{Shaded}

\textbf{For Male;}

\begin{Shaded}
\begin{Highlighting}[]
\NormalTok{  Male\_age }\OtherTok{\textless{}{-}} \FunctionTok{filter}\NormalTok{(data\_Male, Demographics.Question }\SpecialCharTok{==} \StringTok{"Age"}\NormalTok{)}
  \FunctionTok{view}\NormalTok{(Male\_age)}
\end{Highlighting}
\end{Shaded}

\begin{Shaded}
\begin{Highlighting}[]
\NormalTok{  Male\_education }\OtherTok{\textless{}{-}} \FunctionTok{filter}\NormalTok{(data\_Male, Demographics.Question }\SpecialCharTok{==} \StringTok{"Education"}\NormalTok{)}
  \FunctionTok{view}\NormalTok{(Male\_education)}
\end{Highlighting}
\end{Shaded}

\begin{Shaded}
\begin{Highlighting}[]
\NormalTok{  Male\_Employment }\OtherTok{\textless{}{-}} \FunctionTok{filter}\NormalTok{(data\_Male, Demographics.Question }\SpecialCharTok{==} \StringTok{"Employment"}\NormalTok{)}
  \FunctionTok{view}\NormalTok{(Male\_Employment)}
\end{Highlighting}
\end{Shaded}

\begin{Shaded}
\begin{Highlighting}[]
\NormalTok{  Male\_Marital }\OtherTok{\textless{}{-}} \FunctionTok{filter}\NormalTok{(data\_Male, Demographics.Question }\SpecialCharTok{==} \StringTok{"Marital status"}\NormalTok{)}
  \FunctionTok{view}\NormalTok{(Male\_Marital)}
\end{Highlighting}
\end{Shaded}

\begin{Shaded}
\begin{Highlighting}[]
\NormalTok{  Male\_Residence }\OtherTok{\textless{}{-}} \FunctionTok{filter}\NormalTok{(data\_Male, Demographics.Question }\SpecialCharTok{==} \StringTok{"Residence"}\NormalTok{)}
  \FunctionTok{view}\NormalTok{(Male\_Residence)}
\end{Highlighting}
\end{Shaded}

\hypertarget{future-work}{%
\subsubsection{Future Work:}\label{future-work}}

We will visualize the datasets that we split from our main data, using
the value column. Value column make us understand that the \% of people
surveyed in the relevant group who agree with the question (e.g.~the
percentage of women aged 15-24 in Afghanistan who agree that a husband
is justified in hitting or beating his wife if she burns the food)

We may merge the advanced economies data frame ( that we were given in
the previous homework ) into our filtered Int\_women\_day data set. Then
we will analyze the values corresponding to a specific question by
comparing with the average of that value in both developed/emerging
countries. That process of merging data frames and taking the average of
values may not be easy to do so we may want to analyze our values,
genders, questions etc. in a specific country (e.g.~Turkey).

\newpage

\hypertarget{references}{%
\section{REFERENCES}\label{references}}

\textbf{DATA :}

\url{https://data.world/login?next=\%2Fmakeovermonday\%2F2020w10\%2Fworkspace\%2Ffile\%3Ffilename\%3D20200306\%2BData\%2BInternational\%2BWomen\%2527s\%2BDay\%2BViz5\%2BLaunch.csv}

\end{document}
